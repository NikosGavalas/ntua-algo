\documentclass[12pt,a4paper]{article}

\usepackage{geometry}
\geometry{
  a4paper,
  total={170mm,257mm},
  left=20mm,
  right=20mm,
  top=20mm
}

\usepackage[utf8]{inputenc}
\usepackage[english, greek]{babel}
\usepackage[LGR, T1]{fontenc} 

\usepackage{amsmath}

\usepackage{tikz}

\usepackage{titlesec}
\titleformat{\section}{\large}{}{0em}{\textsc}[\titlerule]
\titleformat{\subsection}{\large}{}{0em}{\textbf}[]
\titleformat{\subsubsection}{}{}{0em}{\textit}[]

\title{Αλγόριθμοι και Πολυπλοκότητα}
\author{Γαβαλάς Νίκος, AM 03113121}
\date{Νοέμβριος 2018}

\begin{document}

  \maketitle

  \begin{center}
    \Large{2η Γραπτή Σειρά Ασκήσεων}
  \end{center}

  \section{Άσκηση 1}

  \subsection{α}
  
  \subsubsection{α.1}

  Κανένα από τα τρία άπληστα κριτήρια δεν οδηγούν σε βέλτιστη
  λύση. Παρακάτω δίδονται ένα αντιπαράδειγμα για κάθε κριτήριο: \\
  \\
  \underline{Λιγότερες Επικαλύψεις:}
  \\
  \begin{center}
    \begin{tikzpicture}
      \draw (-2.75, 0) -- node[above] {1} (-1.75, 0);
      \draw (-1.25, 0) -- node[above] {2} (-0.25, 0);
      \draw (0.25, 0) -- node[above] {3} (1.25, 0);
      \draw (1.75, 0) -- node[above] {4} (2.75, 0);

      \draw (-2, 1) -- node[above] {5} (-1, 1);
      \draw (-0.5, 1) -- node[above] {6} (0.5, 1);
      \draw (1, 1) -- node[above] {7} (2, 1);
      
      \draw (-2, 2) -- node[above] {8} (-1, 2);
      \draw (1, 2) -- node[above] {9} (2, 2);

      \draw (-2, 3) -- node[above] {10} (-1, 3);
      \draw (1, 3) -- node[above] {11} (2, 3);
    \end{tikzpicture}
  \end{center}
  Εδώ βέλτιστη επιλογή είναι \( |\{1, 2, 3, 4\}| = 4 \) αλλά
  επιλέγεται η \( |\{6, 1, 4\}| = 3 \) \\
  \\
  \underline{Μεγαλύτερη Διάρκεια:}\\
  \\
  \begin{center}
    \begin{tikzpicture}
      \draw (-0.5, 1) -- node[above] {1} (1, 1);
      \draw (-1.25, 0) -- node[above] {2} (-0.25, 0);
      \draw (0.25, 0) -- node[above] {3} (3.25, 0);
    \end{tikzpicture}
  \end{center}
  Εδώ βέλτιστη είναι η \( |\{2, 3\}| = 2 \) αλλά επιλέγεται η \( |\{2\}| = 1 \) \\
  \\
  \underline{Περισσότερες Επικαλύψεις:} \\
  \\
  \begin{center}
    \begin{tikzpicture}
      \draw (-2.75, 0) -- node[above] {1} (-1.75, 0);
      \draw (-1.25, 0) -- node[above] {2} (-0.25, 0);
      \draw (0.25, 0) -- node[above] {3} (1.25, 0);
      \draw (-2, 1) -- node[above] {4} (-1, 1);
      \draw (-0.5, 1) -- node[above] {5} (0.5, 1);
      \draw (-0.5, 2) -- node[above] {6} (0.5, 2);
      \draw (-0.5, 3) -- node[above] {7} (0.5, 3);
      \draw (-2, 2) -- node[above] {8} (-1, 2);
      \draw (-2, 3) -- node[above] {9} (-1, 3);
    \end{tikzpicture}
  \end{center}
  Εδώ βέλτιστη είναι η \( |\{ 1, 2, 3\}| = 3 \) αλλά επιλέγεται κάποια σαν 
  την \( |\{ 1, 3 \}| = 2 \)

  \subsubsection{α.2}

  

  \subsection{β}
  
  \section{Άσκηση 2}
  
  \section{Άσκηση 3}
  
  \section{Άσκηση 4}
  
  \section{Άσκηση 5}

    Έχουμε \( n \) κεραίες που μπορούν να  λειτουργούν ως πομποί ή δέκτες, καταναλώνοντας αντίστοιχα 
    ενέργεια \( (T_i, R_i) \), με \( R_i \le T_i \), και θέλουμε να τις χωρίσουμε όλες σε
    ζευγάρια με τρόπο τέτοιο ώστε η συνολική κατανάλωση του δικτύου να είναι η 
    ελάχιστη δυνατή.\\
    \\
    Θα ακολουθήσουμε μια άπληστη προσέγγιση: Έστω ότι μέχρι 
    την \( i \)-οστή κεραία έχουμε βέλτιστη λύση. Πάμε να αποφασίσουμε ως τι θα 
    λειτουργεί η επόμενη κεραία. Αν είναι το \( i \) περιττός, τότε απλά κάνουμε 
    την κεραία \( i + 1 \) δέκτη, και προσθέτουμε το \( R_i \) στην συνολική κατανάλωση. 
    Αν είναι το \( i \) άρτιος, τότε κάνουμε την \( i + 1 \) κεραία δέκτη (και 
    προσθέτουμε \( R_i \) στη συνολική κατανάλωση), αλλά αυτή τη φορά πρέπει 
    κάποιος δέκτης από το σύνολο $\{1,...,i+1\}$ να γίνει πομπός, γιατί πρέπει 
    συνολικά τα ζεύγη να μας βγουν $n/2$. Ποιος όμως θα πρέπει να γίνει πομπός;
    Εκείνος που έχει το μικρότερο κόστος για να γίνει, ώστε να εξασφαλίσουμε ότι
    η συνολική κατανάλωση του δικτύου θα είναι ελάχιστη. Το μικρότερο κόστος για
    να γίνει από δέκτης \( \rightarrow \) πομπός μία κεραία, είναι η μικρότερη διαφορά
    $T_i - R_i$, την οποία μπορούμε να παρακολουθούμε με ένα {\latintext min-heap} που σε 
    κάθε βήμα κάνει χρόνο \( O(\log{n}) \) για αυτή τη δουλειά. Οπότε, αφού περάσουμε 
    και από τις $n$ κεραίες, ο αλγόριθμος μας θα κάνει συνολικά \( O(\log{n}) \) για 
    κάθε κεραία, οπότε θα κάνει συνολικά χρόνο \( O(n\log{n}) \).

\end{document}
