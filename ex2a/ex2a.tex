\documentclass[12pt,a4paper]{article}

\usepackage{geometry}
\geometry{
  a4paper,
  total={170mm,257mm},
  left=20mm,
  right=20mm,
  top=20mm
}

\usepackage[utf8]{inputenc}
\usepackage[english, greek]{babel}
\usepackage[LGR, T1]{fontenc} 

\usepackage{amsmath}

\usepackage{titlesec}
\titleformat{\section}{\large}{}{0em}{\textsc}[\titlerule]
\titleformat{\subsection}{\large}{}{0em}{\textbf}[]
\titleformat{\subsubsection}{}{}{0em}{\textit}[]

\title{Αλγόριθμοι και Πολυπλοκότητα}
\author{Γαβαλάς Νίκος, AM 03113121}
\date{Νοέμβριος 2018}

\begin{document}

  % \maketitle

  \begin{center}
    \Large{2η Γραπτή Σειρά Ασκήσεων}
  \end{center}

  \section{Άσκηση 1}

  \subsection{α}
  
  \subsubsection{α.1}

  \subsubsection{α.2}

  \subsection{β}
  
  \section{Άσκηση 2}
  
  \section{Άσκηση 3}
  
  \section{Άσκηση 4}
  
  \section{Άσκηση 5}
    Έχουμε τώρα το ίδιο πρόβλημα με πριν με την προσθήκη ότι οι $n$ κεραίες 
    μπορούν να  λειτουργούν ως πομποί ή δέκτες, καταναλώνοντας αντίστοιχα 
    ενέργεια \( (T_i,R_i) \), με \( R_i \le T_i \), και θέλουμε να τις χωρίσουμε όλες σε
    ζευγάρια με τρόπο τέτοιο ώστε η συνολική κατανάλωση του δικτύου να είναι η 
    ελάχιστη δυνατή. Θα ακολουθήσουμε μια άπληστη προσέγγιση: Έστω ότι μέχρι 
    την \( i \)-οστή κεραία έχουμε βέλτιστη λύση. Πάμε να αποφασίσουμε ως τι θα 
    λειτουργεί η επόμενη κεραία. Αν είναι το \( i \) περιττός, τότε απλά κάνουμε 
    την κεραία \( i + 1 \) δέκτη, και προσθέτουμε το \( R_i \) στην συνολική κατανάλωση. 
    Αν είναι το \( i \) άρτιος, τότε κάνουμε την \( i + 1 \) κεραία δέκτη (και 
    προσθέτουμε \( R_i \) στη συνολική κατανάλωση), αλλά αυτή τη φορά πρέπει 
    κάποιος δέκτης από το σύνολο $\{1,...,i+1\}$ να γίνει πομπός, γιατί πρέπει 
    συνολικά τα ζεύγη να μας βγουν $n/2$. Ποιος όμως θα πρέπει να γίνει πομπός;
    Εκείνος που έχει το μικρότερο κόστος για να γίνει, ώστε να εξασφαλίσουμε ότι
    η συνολική κατανάλωση του δικτύου θα είναι ελάχιστη. Το μικρότερο κόστος για
    να γίνει από δέκτης$\rightarrow$πομπός μία κεραία, είναι η μικρότερη διαφορά
    $T_i - R_i$, την οποία μπορούμε να παρακολουθούμε με ένα min-heap που σε 
    κάθε βήμα κάνει χρόνο $O(logn)$ για αυτή τη δουλειά. Οπότε, αφού περάσουμε 
    και από τις $n$ κεραίες, ο αλγόριθμος μας θα κάνει συνολικά $O(logn)$ σε 
    κάθε κεραία, οπότε θα κάνει συνολικά χρόνο $O(nlogn)$.

\end{document}
