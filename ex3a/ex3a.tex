\documentclass[12pt,a4paper]{article}

\usepackage{geometry}
\geometry{
  a4paper,
  total={170mm,257mm},
  left=20mm,
  right=20mm,
  top=20mm
}

\usepackage[utf8]{inputenc}
\usepackage[english, greek]{babel}
\usepackage[LGR, T1]{fontenc}

\usepackage{amsmath}
\usepackage{amsfonts}

\usepackage{tikz}

\usepackage{titlesec}
\titleformat{\section}{\large}{}{0em}{\textsc}[\titlerule]
\titleformat{\subsection}{\large}{}{0em}{\textbf}[]
\titleformat{\subsubsection}{}{}{0em}{\textit}[]

\title{Αλγόριθμοι και Πολυπλοκότητα}
\author{Γαβαλάς Νίκος, AM 03113121}
\date{Δεκέμβριος 2018}

\begin{document}

  \maketitle

  \begin{center}
    \Large{3η Γραπτή Σειρά Ασκήσεων}
  \end{center}

  \section{Άσκηση 1}

  \section{Άσκηση 2}

  Έχουμε γράφο \( G(V, E) \) με \( p(u) \in \mathbb{N}, u \in V \) και ζητάμε 
  την τιμή  \( c(u), \forall u \in V \), όπου \( c(u) \) είναι η τιμή της 
  φθηνότερης κορυφής που είναι προσπελάσιμη από τη \( u \), μαζί με τη \( u \).

  \subsection{α}

  Aν ο γράφος \( G \) είναι {\latintext DAG}, γνωρίζουμε πως τότε ορίζεται
  τοπολογική διάταξη μεταξύ των κορυφών του, η οποία σχηματίζεται σε χρόνο
  \( Ο(|V| + |E|) \) (γραμμικό).
  \\
  \\
  Διατάσσουμε λοιπόν τις κορυφές και ύστερα,
  προσπελαύνοντας τες αντίστροφα, κρατάμε για κάθε μία τη μικρότερη τιμή
  \( c(u) \) βάσει της σχέσης \( c(u) = \min\{ p(u), \min_{ v : (u, v) \in E}
  \{c(v)\}\} \).
  \\
  \\
  Το δεύτερο βήμα απαιτεί επίσης γραμμικό χρόνο, επομένως ο αλγόριθμος είναι
  γραμμικού χρόνου.

  \subsection{β}

  Γνωρίζουμε ότι κάθε γράφος μπορεί να παρασταθεί ως {\latintext DAG} των 
  {\latintext SCC} του, σε γραμμικό χρόνο.
  \\
  \\
  Κάθε κορυφή \( u \in W \), \( W \) {\latintext SCC} του \( G \),
  θα έχει την ίδια τιμή \( c(u) \) με τις υπόλοιπες κορυφές που
  ανήκουν στην ίδια {\latintext SCC}, αφού ακριβώς επειδή είναι ισχυρά συνεκτική
  συνιστώσα και άρα υπάρχει διαδρομή από και προς οποιοδήποτε
  \( u, v \in W \), αρκεί να
  βρούμε την ελάχιστη \( p(u) \) μεταξύ όλων αυτών και ύστερα σε όλες θα θέσουμε
  \( c(u) \) ίση με αυτήν.
  \\
  \\
  Αρχικά, βρίκουμε λοιπόν τις {\latintext SCC}, και για κάθε μία \( W \) από
  αυτές υπολογίζουμε την ελάχιστη τιμή μεταξύ όλων των κορυφών \( u \in W \),
  δηλαδή το \( P(W) = \min_{u \in W}\{ p(u) \}\).
  \\
  \\
  Στη συνέχεια, σχηματίζουμε τον μεταγράφο που αποτελεί το {\latintext DAG} των 
  {\latintext SCC \( W \)} του \( G \), και τρέχουμε τον αλγόριθμο του
  ερωτήματος (α) υπολογίζοντας τις τιμές \( C(W) \) με βάση τα \( P(W) \).
  \\
  \\
  Tέλος, για κάθε {\latintext SCC \(W\)}, για κάθε κορυφή \(u \in W \), θέτουμε
  \( c(u) = C(W) \).
  \\
  \\
  Η χρονική πολυπλοκότητα του αλγορίθμου είναι γραμμική, αφού κάθε βήμα απαιτεί
  χρόνο \( Ο(|V| + |E|) \).

  \section{Άσκηση 3}
  
  \section{Άσκηση 4}

  Έστω συνεκτικός, μη κατευθυνόμενος γράφος \( G(V, E) \) με \( |V| = n,
  |E| = m \).

  \subsection{α}
  
  \subsection{β}

  \subsection{γ}

  \section{Άσκηση 5}

  \subsection{α}
  
  \subsection{β}
  
  \subsection{γ}

  \subsection{δ}

\end{document}
